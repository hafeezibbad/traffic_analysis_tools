\monitorlab dataset was collected at two laboratories located at Northeastern University, (US) (US-testbed) and Imperial College (UK) (UK-TESTBED). 
This dataset was collected between September 2018 and February 2019. 
A total of 81 devices were used to collect the dataset in 34,586 automated and manually controlled experiments.
The dataset is available on request from Monitor lab\footnote{Instructures available at \url{https://moniotrlab.ccis.neu.edu/imc19/}}.
Further details about this dataset are available in the original paper~\cite{DBLP:conf/imc/RenDCMKH19}.


\subsection{Devices}
\label{sec:monitorlab_devices}
A total of 81 devices used in \monitorlab dataset can be categorized in 6 categories, listed in Table 1 \url{https://moniotrlab.ccis.neu.edu/wp-content/uploads/2019/09/ren-imc19.pdf}. 

\subsection{Experiments}
\label{sec:monitorlab_experiments}
Out of the total 34,586 experiments performed in \monitorlab dataset, 20,777 were performed in US testbed, and 13,809 were performed at UK testbed. 
These experiments include
\begin{itemize}
\item \textbf{Power experiments}:
A total of 487 power experiments were conducted to study the difference between network traffic generated by IoT devices when they are powered-on versus powered-off. 
\item \textbf{Interaction experiments}:
These experiments required interaction with an IoT device. 
The data collection starts when the device powers-on, and finishes 10-15 seconds after the interaction has finished. 
The interaction considered in these experiments include
\begin{itemize}
\item \textit{local action}: User physically interactes with device using voice commands or on-device controls. 
\item \textit{LAN app action}: User interacts with the device using companion smartphone/tablet application, while the smartphone/tablet is connected to local network. 
\item \textit{LAN app action}: User interacts with the device using companion smartphone/tablet application, while the smartphone/tablet is connected to another network. 
\end{itemize}
32,030 automated experiments were run by automating actions through voice commands, or companion app. 2,069 experiements were performed using manual interaction. 


\item \textbf{Idle experiments}:
Idle experiments are those were a device was left powered on, and no interactions happened with the device. Typically, these idle periods averaged at 8 hours per night, once a week. 

\item \textbf{Uncontrolled experiments}:
In these experiments, 36 participants from a user study visited the lab an interacted with IoT devices without following any script. 
These participants accessed the lab 20-30 times a day, with at least one interaction per device. 
During this time, no interactions were triggered through companion apps. 
\end{itemize}

In some cases, the researchers used a VPN from UK lab and US lab to change the exit node of traffic from each of the labs. 

\subsection{Summary from dataset}
\label{sec:monitorlab_summary}
The dataset obtained from \monitorlab contains 37,744 pcap files, totalling to 6.1 GB of network traces.
These traces are divided into 4 sub categories, listed in Table~\ref{tbl:monitorlab_traces_category}

\begin{table}
\begin{center}
\begin{tabular}{c c c c}
\toprule
Category & Number of devices  & Number of traces & volume \\
\midrule
UK & 35 & 8023 & 2.58 GB \\
UK-VPN &  35 & 9175 & 2.21 GB \\
US & 46 & 10324  & 2.43 GB \\
US-VPN & 46 &  10222 & 1.92 GB\\
\bottomrule
\end{tabluar}
\caption{Sub categories of network traces available in \monitorlab dataset}
\label{tbl:monitorlab_traces_category}
\end{center}
\end{table}